\begin{abstract} Semantic correspondence aims to establish pixel-level matches between semantically similar object parts across different images, a task complicated by variations in viewpoint, scale, and domain. Recent Visual Foundation Models like DINO and SAM have demonstrated rich internal representations that offer a powerful basis for dense matching without explicit supervision. This work investigates efficient adaptation strategies for Visual Foundation Models (DINOv2, DINOv3, SAM) on the SPair-71k benchmark. We evaluate three distinct approaches: a training-free baseline, a light fine-tuning of the last layers (Linear Probing), and Low-Rank Adaptation (LoRA) applied to attention mechanisms. To refine spatial precision, we implement a window soft-argmax mechanism replacing the standard argmax. Our experiments demonstrate that light fine-tuning of the last layers significantly outperforms both the pre-trained backbone baseline and the LoRA approach. While the training-free baseline establishes a solid lower bound, and LoRA offers theoretical flexibility, we find that targeted linear probing effectively adapts the model without distorting the robust pre-trained feature manifold, achieving the highest PCK accuracy across multiple thresholds. \end{abstract}